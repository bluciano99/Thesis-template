%**************************************************************
% file contenente le impostazioni della tesi
%**************************************************************

%**************************************************************
% Frontespizio
%**************************************************************

% Autore
\newcommand{\myName}{Luciano Wu}
\newcommand{\myTitle}{Titolo della tesi}

% Tipo di tesi                   
\newcommand{\myDegree}{Tesi di laurea}

% Università             
\newcommand{\myUni}{Università degli Studi di Padova}

% Facoltà       
\newcommand{\myFaculty}{Corso di Laurea in Informatica}

% Dipartimento
\newcommand{\myDepartment}{Dipartimento di Matematica "Tullio Levi-Civita"}

% Titolo del relatore
\newcommand{\profTitle}{Prof.}

% Relatore
\newcommand{\myProf}{Francesco Ranzato}

% Luogo
\newcommand{\myLocation}{Padova}

% Anno accademico
\newcommand{\myAA}{2021-2022}

% Data discussione
\newcommand{\myTime}{Luglio 2022}


%**************************************************************
% Impostazioni di impaginazione
% see: http://wwwcdf.pd.infn.it/AppuntiLinux/a2547.htm
%**************************************************************

\setlength{\parindent}{14pt}   % larghezza rientro della prima riga
\setlength{\parskip}{0pt}   % distanza tra i paragrafi


%**************************************************************
% Impostazioni di biblatex
%**************************************************************
\bibliography{bibliografia} % database di biblatex 

\defbibheading{bibliography} {
    \cleardoublepage
    \phantomsection 
    \addcontentsline{toc}{chapter}{\bibname}
    \chapter*{\bibname\markboth{\bibname}{\bibname}}
}

\setlength\bibitemsep{1.5\itemsep} % spazio tra entry

\DeclareBibliographyCategory{opere}
\DeclareBibliographyCategory{web}

\addtocategory{opere}{womak:lean-thinking}
\addtocategory{web}{site:agile-manifesto}

\defbibheading{opere}{\section*{Riferimenti bibliografici}}
\defbibheading{web}{\section*{Siti Web consultati}}


%**************************************************************
% Impostazioni di caption
%**************************************************************
\captionsetup{
    tableposition=top,
    figureposition=bottom,
    font=small,
    format=hang,
    labelfont=bf
}

%**************************************************************
% Impostazioni di glossaries
%**************************************************************
\makeglossaries

%**************************************************************
% Acronimi
%**************************************************************


\newacronym[description={\glslink{apig}{Application Program Interface}}]
    {api}{API}{Application Program Interface}
\newacronym[description={\glslink{urlg}{Uniform Resource Locator}}]
    {url}{URL}{Uniform Resource Locator}
\newacronym[description={\glslink{voipg}{Voice over Internet Protocol}}]
    {VoiP}{Voice over Internet Protocol}
\newacronym[description={\glslink{restg}{Representational state transfer}}]
    {restg}{Representational state transfer}



%**************************************************************
% Glossario
%**************************************************************
%\renewcommand{\glossaryname}{Glossario}

\newglossaryentry{restg}
{
    name=\glslink{restg}{REST},
    text=REST,
    sort=REST,
    description={è uno stile architetturale per sistemi distribuiti. L'espressione "representational state transfer" e il suo acronimo, REST, fu introdotto nel 2000 nella tesi di dottorato di Roy Fielding e vennero rapidamente adottati dalla comunità di sviluppatori Internet. I metodi più utilizzati sono: GET, POST, PUT, PATCH e DELETE}
}
\newglossaryentry{gitg}{
    name=\glslink{gitg}{Git},
    text=Git,
    sort=git,
    description={un software per il controllo di versionamento per lo sviluppo delle applicazioni, nato nel 2005. Permette di creare rami e unire i rami, per un grande progetto che deve iniziare un implementazione di una nuova funzione si può creare un ramo feature e lavorare su questo ramo per mantenere il corretto funzionamento del programma nel ramo main, una volta che la feature è tutta finita lo si può fare il merge nel main del progetto}}

\newglossaryentry{apig}
{
    name=\glslink{apig}{API},
    text=API,
    sort=api,
    description={in informatica con il termine \emph{Application Programming Interface API} (ing. interfaccia di programmazione di un'applicazione) si indica ogni insieme di procedure disponibili al programmatore, di solito raggruppate a formare un set di strumenti specifici per l'espletamento di un determinato compito all'interno di un certo programma. La finalità è ottenere un'astrazione, di solito tra l'hardware e il programmatore o tra software a basso e quello ad alto livello semplificando così il lavoro di programmazione}
}

\newglossaryentry{urlg}
{
    name=\glslink{urlg}{url},
    text=URL,
    sort=URL,
    description={è una sequenza di caratteri che identifica univocamente l'indirizzo di una risorsa su una rete di computer, come ad esempio una pagina web, tipicamente presente su un host server e resa accessibile a un client. Un esempio di URL famoso può essere 'www.google.com'
    }
}
\newglossaryentry{responsiveg}
{
    name=\glslink{responsiveg}{responsive},
    text=responsive,
    sort=responsive,
    description={indica una tecnica di web design per la realizzazione di siti in grado di adattarsi graficamente in modo automatico al dispositivo coi quali vengono visualizzati, come computer, cellulare, monitor e TV, riducendo al minimo la necessità dell'utente di ridimensionare e scorrere i contenuti
    }
}
\newglossaryentry{mascherag}
{
    name=\glslink{mascherag}{maschera},
    text=maschera,
    sort=maschera,
    description={indica interfaccia che viene viene visualizzato all'utente}
}
\newglossaryentry{backendg}
{
    name=\glslink{backendg}{back-end},
    text=back-end,
    sort=back-end,
    description={parte del software che elabora i dati generati dal front-end e comunica con il data-base per scrive e leggere dati}
}
\newglossaryentry{frontendg}
{
    name=\glslink{frontendg}{front-end},
    text=front-end,
    sort=front-end,
    description={è la parte di un sistema software che gestisce l'interazione con l'utente o con sistemi esterni che producono dati di ingresso, comuni con il back-end tramite le chiamate REST}
}
% \newglossaryentry{asd}
% {
%     name=\glslink{asd},
%     text=asd,
%     sort=asd,
%     description={}
% }

\newglossaryentry{voipg}{
    name=\glslink{voipg}{VoiP},
    text=VoiP,
    sort=Voip,
    description={in telecomunicazioni e informatica, indica una tecnologia che rende possibile effettuare una conversazione, analoga a quella che si potrebbe ottenere con una rete telefonica, sfruttando una connessione Internet o una qualsiasi altra rete di telecomunicazioni dedicata a commutazione di pacchetto, che utilizzi il protocollo IP senza connessione per il trasporto dati}}



\newglossaryentry{Springg}{
    name=\glslink{Springg}{Spring},
    text=Spring,
    sort=spring,
    description={un framework open source per lo sviluppo delle applicazioni su piattaforma Java, nato nel 2002. Le sue pricipali punti di forza sono: flessibilità ,modularità, elevata testabilità e una grande community. Per fare questo spring si basa sui seguenti principi: dependency injection}}

 % database di termini


%**************************************************************
% Impostazioni di graphicx
%**************************************************************
\graphicspath{{immagini/}} % cartella dove sono riposte le immagini


%**************************************************************
% Impostazioni di hyperref
%**************************************************************
\hypersetup{
    %hyperfootnotes=false,
    %pdfpagelabels,
    %draft,	% = elimina tutti i link (utile per stampe in bianco e nero)
    colorlinks=true,
    linktocpage=true,
    pdfstartpage=1,
    pdfstartview=,
    % decommenta la riga seguente per avere link in nero (per esempio per la stampa in bianco e nero)
    %colorlinks=false, linktocpage=false, pdfborder={0 0 0}, pdfstartpage=1, pdfstartview=FitV,
    breaklinks=true,
    pdfpagemode=UseNone,
    pageanchor=true,
    pdfpagemode=UseOutlines,
    plainpages=false,
    bookmarksnumbered,
    bookmarksopen=true,
    bookmarksopenlevel=1,
    hypertexnames=true,
    pdfhighlight=/O,
    %nesting=true,
    %frenchlinks,
    urlcolor=webbrown,
    linkcolor=RoyalBlue,
    citecolor=webgreen,
    %pagecolor=RoyalBlue,
    %urlcolor=Black, linkcolor=Black, citecolor=Black, %pagecolor=Black,
    pdftitle={\myTitle},
    pdfauthor={\textcopyright\ \myName, \myUni, \myFaculty},
    pdfsubject={},
    pdfkeywords={},
    pdfcreator={pdfLaTeX},
    pdfproducer={LaTeX}
}

%**************************************************************
% Impostazioni di itemize
%**************************************************************
% \renewcommand{\labelitemi}{$\ast$}

%\renewcommand{\labelitemi}{$\bullet$}
%\renewcommand{\labelitemii}{$\cdot$}
%\renewcommand{\labelitemiii}{$\diamond$}
%\renewcommand{\labelitemiv}{$\ast$}


%**************************************************************
% Impostazioni di listings
%**************************************************************
\lstset{
    language=[LaTeX]Tex,%C++,
    keywordstyle=\color{RoyalBlue}, %\bfseries,
    basicstyle=\small\ttfamily,
    %identifierstyle=\color{NavyBlue},
    commentstyle=\color{Green}\ttfamily,
    stringstyle=\rmfamily,
    numbers=none, %left,%
    numberstyle=\scriptsize, %\tiny
    stepnumber=5,
    numbersep=8pt,
    showstringspaces=false,
    breaklines=true,
    frameround=ftff,
    frame=single
} 


%**************************************************************
% Impostazioni di xcolor
%**************************************************************
\definecolor{webgreen}{rgb}{0,.5,0}
\definecolor{webbrown}{rgb}{.6,0,0}


%**************************************************************
% Altro
%**************************************************************

\newcommand{\omissis}{[\dots\negthinspace]} % produce [...]

% eccezioni all'algoritmo di sillabazione
\hyphenation
{
    ma-cro-istru-zio-ne
    gi-ral-din
}

\newcommand{\sectionname}{sezione}
\addto\captionsitalian{\renewcommand{\figurename}{Figura}
                       \renewcommand{\tablename}{Tabella}}

\newcommand{\glsfirstoccur}{\ap{{[g]}}}

\newcommand{\intro}[1]{\emph{\textsf{#1}}}

%**************************************************************
% Environment per ``rischi''
%**************************************************************
\newcounter{riskcounter}                % define a counter
\setcounter{riskcounter}{0}             % set the counter to some initial value

%%%% Parameters
% #1: Title
\newenvironment{risk}[1]{
    \refstepcounter{riskcounter}        % increment counter
    \par \noindent                      % start new paragraph
    \textbf{\arabic{riskcounter}. #1}   % display the title before the 
                                        % content of the environment is displayed 
}{
    \par\medskip
}

\newcommand{\riskname}{Rischio}

\newcommand{\riskdescription}[1]{\textbf{\\Descrizione:} #1.}

\newcommand{\risksolution}[1]{\textbf{\\Soluzione:} #1.}

%**************************************************************
% Environment per ``use case''
%**************************************************************
\newcounter{usecasecounter}             % define a counter
\setcounter{usecasecounter}{0}          % set the counter to some initial value

%%%% Parameters
% #1: ID
% #2: Nome
\newenvironment{usecase}[2]{
    \renewcommand{\theusecasecounter}{\usecasename #1}  % this is where the display of 
                                                        % the counter is overwritten/modified
    \refstepcounter{usecasecounter}             % increment counter
    \vspace{10pt}
    \par \noindent                              % start new paragraph
    {\large \textbf{\usecasename #1: #2}}       % display the title before the 
                                                % content of the environment is displayed 
    \medskip
}{
    \medskip
}

\newcommand{\usecasename}{UC}
\newcommand{\gl}{\textsuperscript{G}}
\newcommand{\usecaseactors}[1]{\textbf{\\Attori Principali:} #1. \vspace{4pt}}
\newcommand{\usecasepre}[1]{\textbf{\\Precondizioni:} #1. \vspace{4pt}}
\newcommand{\usecasedesc}[1]{\textbf{\\Descrizione:} #1. \vspace{4pt}}
\newcommand{\usecasepost}[1]{\textbf{\\Postcondizioni:} #1. \vspace{4pt}}
\newcommand{\usecasealt}[1]{\textbf{\\Scenario Alternativo:} #1. \vspace{4pt}}

%**************************************************************
% Environment per ``namespace description''
%**************************************************************

\newenvironment{namespacedesc}{
    \vspace{10pt}
    \par \noindent                              % start new paragraph
    \begin{description} 
}{
    \end{description}
    \medskip
}

\newcommand{\classdesc}[2]{\item[\textbf{#1:}] #2}
