
%**************************************************************
% Acronimi
%**************************************************************


\newacronym[description={\glslink{apig}{Application Program Interface}}]
    {api}{API}{Application Program Interface}
\newacronym[description={\glslink{urlg}{Uniform Resource Locator}}]
    {url}{URL}{Uniform Resource Locator}
\newacronym[description={\glslink{voipg}{Voice over Internet Protocol}}]
    {VoiP}{Voice over Internet Protocol}
\newacronym[description={\glslink{restg}{Representational state transfer}}]
    {restg}{Representational state transfer}



%**************************************************************
% Glossario
%**************************************************************
%\renewcommand{\glossaryname}{Glossario}

\newglossaryentry{restg}
{
    name=\glslink{restg}{REST},
    text=REST,
    sort=REST,
    description={è uno stile architetturale per sistemi distribuiti. L'espressione "representational state transfer" e il suo acronimo, REST, fu introdotto nel 2000 nella tesi di dottorato di Roy Fielding e vennero rapidamente adottati dalla comunità di sviluppatori Internet. I metodi più utilizzati sono: GET, POST, PUT, PATCH e DELETE}
}
\newglossaryentry{gitg}{
    name=\glslink{gitg}{Git},
    text=Git,
    sort=git,
    description={un software per il controllo di versionamento per lo sviluppo delle applicazioni, nato nel 2005. Permette di creare rami e unire i rami, per un grande progetto che deve iniziare un implementazione di una nuova funzione si può creare un ramo feature e lavorare su questo ramo per mantenere il corretto funzionamento del programma nel ramo main, una volta che la feature è tutta finita lo si può fare il merge nel main del progetto}}

\newglossaryentry{apig}
{
    name=\glslink{apig}{API},
    text=API,
    sort=api,
    description={in informatica con il termine \emph{Application Programming Interface API} (ing. interfaccia di programmazione di un'applicazione) si indica ogni insieme di procedure disponibili al programmatore, di solito raggruppate a formare un set di strumenti specifici per l'espletamento di un determinato compito all'interno di un certo programma. La finalità è ottenere un'astrazione, di solito tra l'hardware e il programmatore o tra software a basso e quello ad alto livello semplificando così il lavoro di programmazione}
}

\newglossaryentry{urlg}
{
    name=\glslink{urlg}{url},
    text=URL,
    sort=URL,
    description={è una sequenza di caratteri che identifica univocamente l'indirizzo di una risorsa su una rete di computer, come ad esempio una pagina web, tipicamente presente su un host server e resa accessibile a un client. Un esempio di URL famoso può essere 'www.google.com'
    }
}
\newglossaryentry{responsiveg}
{
    name=\glslink{responsiveg}{responsive},
    text=responsive,
    sort=responsive,
    description={indica una tecnica di web design per la realizzazione di siti in grado di adattarsi graficamente in modo automatico al dispositivo coi quali vengono visualizzati, come computer, cellulare, monitor e TV, riducendo al minimo la necessità dell'utente di ridimensionare e scorrere i contenuti
    }
}
\newglossaryentry{mascherag}
{
    name=\glslink{mascherag}{maschera},
    text=maschera,
    sort=maschera,
    description={indica interfaccia che viene viene visualizzato all'utente}
}
\newglossaryentry{backendg}
{
    name=\glslink{backendg}{back-end},
    text=back-end,
    sort=back-end,
    description={parte del software che elabora i dati generati dal front-end e comunica con il data-base per scrive e leggere dati}
}
\newglossaryentry{frontendg}
{
    name=\glslink{frontendg}{front-end},
    text=front-end,
    sort=front-end,
    description={è la parte di un sistema software che gestisce l'interazione con l'utente o con sistemi esterni che producono dati di ingresso, comuni con il back-end tramite le chiamate REST}
}
% \newglossaryentry{asd}
% {
%     name=\glslink{asd},
%     text=asd,
%     sort=asd,
%     description={}
% }

\newglossaryentry{voipg}{
    name=\glslink{voipg}{VoiP},
    text=VoiP,
    sort=Voip,
    description={in telecomunicazioni e informatica, indica una tecnologia che rende possibile effettuare una conversazione, analoga a quella che si potrebbe ottenere con una rete telefonica, sfruttando una connessione Internet o una qualsiasi altra rete di telecomunicazioni dedicata a commutazione di pacchetto, che utilizzi il protocollo IP senza connessione per il trasporto dati}}



\newglossaryentry{Springg}{
    name=\glslink{Springg}{Spring},
    text=Spring,
    sort=spring,
    description={un framework open source per lo sviluppo delle applicazioni su piattaforma Java, nato nel 2002. Le sue pricipali punti di forza sono: flessibilità ,modularità, elevata testabilità e una grande community. Per fare questo spring si basa sui seguenti principi: dependency injection}}

