
%**************************************************************
% Acronimi
%**************************************************************


\newacronym[description={\glslink{apig}{Application Program Interface}}]
    {api}{API}{Application Program Interface}

\newacronym[description={\glslink{umlg}{Unified Modeling Language}}]
    {uml}{UML}{Unified Modeling Language}

%**************************************************************
% Glossario
%**************************************************************
%\renewcommand{\glossaryname}{Glossario}

\newglossaryentry{apig}
{
    name=\glslink{apig}{API},
    text=Application Program Interface,
    sort=api,
    description={in informatica con il termine \emph{Application Programming Interface API} (ing. interfaccia di programmazione di un'applicazione) si indica ogni insieme di procedure disponibili al programmatore, di solito raggruppate a formare un set di strumenti specifici per l'espletamento di un determinato compito all'interno di un certo programma. La finalità è ottenere un'astrazione, di solito tra l'hardware e il programmatore o tra software a basso e quello ad alto livello semplificando così il lavoro di programmazione}
}

\newglossaryentry{umlg}
{
    name=\glslink{uml}{UML},
    text=UML,
    sort=uml,
    description={in ingegneria del software \emph{UML, Unified Modeling Language} (ing. linguaggio di modellazione unificato) è un linguaggio di modellazione e specifica basato sul paradigma object-oriented. L'\emph{UML} svolge un'importantissima funzione di ``lingua franca'' nella comunità della progettazione e programmazione a oggetti. Gran parte della letteratura di settore usa tale linguaggio per descrivere soluzioni analitiche e progettuali in modo sintetico e comprensibile a un vasto pubblico}
}

\newglossaryentry{VoiPg}{name={VoiP},text=voice over Internet Protocol,description={in telecomunicazioni e informatica, indica una tecnologia che rende possibile effettuare una conversazione, analoga a quella che si potrebbe ottenere con una rete telefonica, sfruttando una connessione Internet o una qualsiasi altra rete di telecomunicazioni dedicata a commutazione di pacchetto, che utilizzi il protocollo IP senza connessione per il trasporto Dati.}}

\newglossaryentry{Kanbang}{name={Kanban},description={termine giapponese che letteralmente significa "insegna", indica un elemento del sistema Just in time di reintegrazione delle scorte a mano a mano che vengono consumate.}}

\newglossaryentry{Gitg}{name={Git},description={un software per il controllo di versionamento per lo sviluppo delle applicazioni, nato nel 2005. Permette di creare rami e unire i rami, per un grande progetto che deve iniziare un implementazione di una nuova funzione si può creare un ramo feature e lavorare su questo ramo per mantenere il corretto funzionamento del programma nel ramo main, una volta che la feature è tutta finita lo si può fare il merge nel main del progetto.}}

\newglossaryentry{Springg}{name={spring},description={un framework open source per lo svipullo di applicazioni su piattaforma java, nato nel 2002. Le sue pricipali punti di forza sono: flessibilità ,modularità, elevata testabilità e una grande community. Per fare questo spring si basa sui seguenti principi: dependency injection, }}

% \newglossaryentry{software}{name={software},description={}}

% \newglossaryentry{}{name={spring},description={}}
% \newglossaryentry{}{name={spring},description={}}
% \newglossaryentry{}{name={spring},description={}}
% \newglossaryentry{}{name={spring},description={}}
% \newglossaryentry{}{name={spring},description={}}
