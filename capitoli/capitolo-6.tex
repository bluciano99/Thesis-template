% !TEX encoding = UTF-8
% !TEX TS-program = pdflatex
% !TEX root = ../tesi.tex

%**************************************************************
\chapter{Conclusioni}
\label{cap:conclusioni}
%**************************************************************
\section{Raggiungimento degli obiettivi}
\subsection{Obiettivi fissati}
Tutti gli obiettivi fissati all'inizio dello stage con il tutor aziendale sono stati raggiunti, oltre ai requisiti obbligatori sono stati soddisfatti anche i requisiti desiderabili e i facoltativi. Infatti la web-app fonisce tutte le funzionalità richieste e offre inoltre la possibilità di salvare i piatti nella lista dei preferiti, filtrare il menù rimuovendo i piatti contenenti gli ingredienti presenti nella blacklist e fornire una recensione ad un piatto.
\begin{center}
    \rowcolors{2}{Cyan!10}{GreenYellow!10}
    \renewcommand{\arraystretch}{1.5}
    \begin{longtable}{ |p{1.5cm}|p{9cm}|p{2cm}|  }
        \hline
        \multicolumn{3}{|c|}{Obiettivi fissati} \\
        \hline
        Codice&Descrizione &Esito \\
        \hline
        \endhead
        O01&Acquisizione competenze sulle tematiche sopra descritte&Superato \\
        O02&Capacità di raggiungere gli obiettivi richiesti in autonomia seguendo il cronoprogramma&Superato\\
        O03&Portare a termine le implementazioni previste con una percentuale di superamento pari al 80\%&Superato\\
        D01&Portare a termine le implementazioni previste con una percentuale di superamento pari al 100\%&Superato\\
        F01&Apportare un valore aggiunto al gruppo di lavoro durante le fasi di progettazione delle interfacce&Superato\\
\hline
\caption{\label{tab:tabella raggiungimento obiettivi fissati}Tabella del raggiungimento degli obiettivi fissati.}
\end{longtable}
\end{center}
\pagebreak

\subsection{Svolgimento del lavoro}
I lavori pianificati per le prime settimane di stage sono state svolte senza particolari difficoltà, infatti le attività di studio e formazione sono terminate con un piccolo anticipo rispetto alla pianificazione e il tempo risparmiato viene utilizzato alla fine per implementare le funzionalità aggiuntive. Questo guadagno di tempo è dovuto ad una buona familiarità con i linguaggi di programmazione acquisita durante gli anni di studio nell'ambito dell'informatica.\\
La fase di progettazione ed implementazione delle maschere è risultata più rapida del previsto anche grazie alla mia esperienza pregressa come cameriere in un ristorante sushi.\\
La parte che ha impiegato più ore di quelli previsti è stato la fase di integrazione tra le varie maschere dove bisogna mantenere i dati salvati di ogni maschera, questo problema è stato risolto grazie ai componenti service forniti da Angular. Un'altro strumento che ha dato un grande aiuto è discord dove ci ha permesso di comunicare facilmente con altri collabolatori del progetto.\\
\begin{center}
    \rowcolors{2}{Cyan!10}{GreenYellow!10}
    \renewcommand{\arraystretch}{1.5}
    \begin{longtable}{ |p{3cm}|p{9cm}|  }
        \hline
        Settimana&Attività \\
        \hline
        \endhead
        Prima settimana&\begin{itemize}
            \item Incontro con il tutor aziendale e i collaboratori
            \item Identificazione dei requisiti
            \item Formazione su strumenti di lavoro
            \item Ripasso HTML5,JavaScript e css
        \end{itemize}\\
        Seconda settimana& \begin{itemize}
            \item Formazione su Spring Core
            \item Formazione su Spring Boot
            \item Formazione su linguaggio TypeScript
            \item Formazione su Angular components
        \end{itemize}\\ 
        Terza settimana&\begin{itemize}
            \item Formazione su Angular data biding e pipes
            \item Formazione su Angular routing e forms
            \item Formazione su Angular lifecycle hooks
            \item Analisi dei requisiti
        \end{itemize}\\
        Quarta ettimana&\begin{itemize}
            \item Progettazione delle interfacce tramite Figma
            \item Indentificazione dei colori della web-app
            \item Formazione sulla libreria Angular material
            \item Progettazione maschera d'accesso
        \end{itemize}\\
        Quinta settimana&\begin{itemize}
            \item Progettazione navbar
            \item Progettazione maschera Menù
            \item Progettazione maschera gestione del tavolo
            \item Progettazione maschera gestione degli ordini
            \item Progettazione API
        \end{itemize}\\
        Sesta settimana&\begin{itemize}
            \item Progettazione componenti Service
            \item Realizzazione funzionamento merge ordini
            \item Realizzazione funzionamento sposta ordini
            \item Progettazione componenti Guard
        \end{itemize}\\
        Settima settimana&\begin{itemize}
            \item Integrazione dei componenti
            \item Decorazione finale della web-app
        \end{itemize}\\
        Ottava settimana&\begin{itemize}
            \item Verifica e validazione della web-app
            \item Documentazione analisi tecnica
            \item Collaudo finale
        \end{itemize}\\
\hline
\caption{\label{tab:tabella consuntivo lavoro}Tabella del consuntivo di lavoro.}
\end{longtable}
\end{center}
%**************************************************************
% \section{Consuntivo finale}

%**************************************************************
% \section{Raggiungimento degli obiettivi}

%**************************************************************
\section{Conoscenze acquisite}
Durante lo stage nell'azienda SyncLab, ho acquisito molte nuove conoscenze tecniche.\\
Ho imparato ad usare Angular che è uno dei più famosi framework per lo sviluppo front-end per le applicazioni. L'apprendimento di Angular è stato svolto principalmente in autonomia, consultando vari siti web e video a tema, ottenendo così una maggiore comprensione del framework.\\
Durante la fase di testing ho approfondito il framework Spring di Java al fine di progettare dei mock delle \gls{apig} che rispecchiano il più possibile implementazione del back-end future.\\
Lavorare in Angular mi ha dato la possibilità di imparare bene TypeScript, ciò mi ha permesso di avere una visione più ampia dei problemi che riguardano JavaScript, come il sistema dei tipi e il processo di debugging.\\
Ogni settimana è stato fatto un incontro in sede con il tutor aziendale, dove si è discusso l'avanzamento della web-app, le difficoltà e la ripianificazione della settimana seguente in caso di cambiamenti, ciò mi permesso di familiarizzare con il metodo Agile.\\
% Tutto questo ha consolidato le mie competenze di programmazione comprese durante gli anni di studio e mi ha dato tanti notevoli insegnamenti.
%**************************************************************

% fare come Stafa
\section{Valutazione}
\subsection{L'azienda}
Durante il processo di sviluppo tutte le persone dell'azienda sono state molto disponibili, inizialmente mi hanno fatto illustrato il back-end di un grande progetto fornendomi consigli su come scrive buon codice al fine facilitare il debugging. Il tutor aziendale è sempre stato molto gentile e mi ha dato la possibilità di lavolare sul front-end, che è sempre stato il mio ambito di sviluppo preferito.\\
Infine tutto ciò mi ha permesso di imparare a lavorare con scadenze rigorose come dentro ad un team vero e proprio.\\
\subsection{Personale}
Lo stage mi ha ampliamente soddisfatto, mi ha fatto conoscere meglio l'ambiente di lavoro di un programmatore permettendomi di usare tutte le competenze acquisite durante il corso di studio universitario.\\
Durante il tirocinio ho visto intero processo di sviluppo di un applicativo, partendo da un problema reale arrivando alla implementazione della web-app questo mi è stato particolarmente importante per il mio avanzamento professionale di programmatore.
Arrivando alla fine, intera esperienza dello stage è più che positiva, ritengo che il tirocinio sia molto importante per uno studente, permettendolo di sfruttare e valutare le proprie conoscenze acquisite e particolarmente per capire e riflettere sul percorso di studio scelto, infatti lo stage mi ha affermato di continuare la carriera lavorativa nell'ambito di programmazione.