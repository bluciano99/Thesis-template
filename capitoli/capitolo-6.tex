% !TEX encoding = UTF-8
% !TEX TS-program = pdflatex
% !TEX root = ../tesi.tex

%**************************************************************
\chapter{Conclusioni}
\label{cap:conclusioni}
%**************************************************************
\section{Raggiungimento degli obiettivi}
\subsection{Obiettivi fissati}
Tutti gli obiettivi fissati all'inizio dello con il tutor aziendale sono stati raggiunti, oltre ai requisiti obbligatori sono stati soddisfatti anche i requisiti desiderabili e i facoltativi. Infatti la web-app fonisce tutte le funzionalità richieste e offre inoltre la possibilità di salvare i piatti nella lista dei preferiti, filtrare il menù rimuovendo i piatti contenenti gli ingredienti presenti nella blacklist e fornire una recensione ad un piatto.
\begin{center}
    \rowcolors{2}{Cyan!10}{GreenYellow!10}
    \renewcommand{\arraystretch}{1.5}
    \begin{longtable}{ |p{1.5cm}|p{9cm}|p{2cm}|  }
        \hline
        \multicolumn{3}{|c|}{Obiettivi fissati} \\
        \hline
        Codice&Descrizione &Esito \\
        \hline
        \endhead
        O01&Acquisizione competenze sulle tematiche sopra descritte&Superato \\
        O02&Capacità di raggiungere gli obiettivi richiesti in autonomia seguendo il cronoprogramma&Superato\\
        O03&Portare a termine le implementazioni previste con una percentuale di superamento pari al 80\%&Superato\\
        D01&Portare a termine le implementazioni previste con una percentuale di superamento pari al 100\%&Superato\\
        F01&Apportare un valore aggiunto al gruppo di lavoro durante le fasi di progettazione delle interfacce&Superato\\
\hline
\caption{\label{tab:tabella raggiungimento obiettivi fissati}Tabella raggiungimento degli obiettivi fissati.}
\end{longtable}
\end{center}
\pagebreak

\subsection{svolgimento del lavoro}
I lavori pianificati per le prime settimane di stage sono state svolte senza particolari difficoltà, infatti le attività di studio e formazione sono terminate con un piccolo anticipo rispetto alla pianificazione e il tempo risparmiato viene utilizzato alla fine per implementare le funzionalità aggiuntive. Questo guadagno di tempo è dovuto ad una buona familiarità con i linguaggi di programmazione acquisita durante gli otti anni di studio nell'ambito di informatica.\\
Avendo lavorato nei ristoranti sushi con il ruolo di cameriere, quindi sapendo di preciso i processi che ci sono, la fase di  progettazione ed implementazione delle maschere sono anche loro andati con un pò di anticipo.\\
La parte che ha impiegato più ore di quelli previsti è stato la fase di integrazione tra le varie maschere dove bisogna mantenere i dati salvati di ogni maschera, questo problema è stato risolto grazie ai componenti service forniti da Angular. Un'altro strumento che ha dato un grande aiuto è discord dove ci ha permesso di comunicare facilmente con altri collabolatori del progetto.\\
%**************************************************************
% \section{Consuntivo finale}

%**************************************************************
% \section{Raggiungimento degli obiettivi}

%**************************************************************
\section{Conoscenze acquisite}
Durante lo stage nell'azienda SyncLab, ho acquisito molte nuove tecnologie più utilizzati nel mondo.\\
Ho imparato ad usare Angular che è uno dei più famosi framework per lo sviluppo front-end per le applicazioni, progettare il front-end è sempre stato il mio ambito più amato. Apprendimento di Angular è tutto svolto in autonomia, per fare questo ho dovuto consultare vari siti web e video di tutorial su YouTube, così mi ha fatto crescere la mia capacità di imparare nuovi skills da solo.\\
Per l'implementazione delle maschere il css è uno dei linguaggi più importanti, Angular ha una buona integrazione con la sintassi SCSS, infatti, ho dovuto imparare ad utilizzare questa nuova estensione.\\
La progettazione delle API tramite Stoplight mi ha fatto apprendere spring, che sarà il back-end della web-app quindi per avere una buona integrazione tra i due componenti bisogna avere una buona conoscenza di spring.\\
Lavorando con Angular mi ha dato la possibilità di imparare bene TypeScript, utilizzando TypeScript mi ha fatto notare i problemi che ha JavaScript, tramite TypeScript tutti gli errori sintaticci vengono rilevati prima della compilazione perciò mi ha risparmiato molto tempo per il debugging.\\
Ogni settimana viene fatto un incontro in sede con il tutor, dove si discute l'avanzamento della web-app, se ci sono stati delle difficoltà e viene ripianificato il lavoro della settimana seguente in caso di cambiamenti, svolgendo in questo modo lo stage mi ha fatto familiarizzare con il metodo Agile.\\
Infine il tirocinio mi è stato molto fondamente per conoscere il mondo del lavoro dell'informatica, difatti SyncLab è uno delle aziende molto importante nell'ambito IT. Non solo questo, lo stage mi ha anche fatto vedere come si è cambiato il lavoro dopo la pandemia covid-19, tanti lavori adesso possono svolti in remoto e tutto questo è possibile grazie alle nuove applicazioni sviluppate da programmatori, ciò mi ha fatto capire importanza dell'informatica.\\
Tutto questo mi ha consilidato le mie competenze di programmazione comprese durante gli anni di studio e mi ha dato tanti notevoli insegnamenti.
%**************************************************************

% fare come Stafa
\section{Valutazione personale}
Lo stage mi ha ampliamente soddisfatto. Mi ha fatto conoscere meglio i miei limiti ma soprattutto il reale lavoro di programmatore, in effetti durante le 320 ore di tirocinio dove si lavora come un vero e proprio dipendente 8 ore al giorno, mi ha messo in prova dove posso usare tutte le competenze acquisite durante il corso di studio universitario. Ho visto intero processo di sviluppo di un software. Si è partito da un problema reale che si trova nei ristoranti sushi, si è analizzato tutti i requisiti che servono per risolverlo infine si è arrivato alla progettazione e implementazione del programma.\\
Durante il processo di sviluppo, dove si è svolto il progetto con il tutor e un altro studente, mi ha rafforzato le capacità di lavoro in gruppo, in particolare nell'aspetto organizzativo come ad esempio il ordini.service, che è utilizzato in più maschere dell'applicazione, ho dovuto iniziarlo prima, così facendo il mio compagno lo poteva utilizzare e testare il suo coretto funzionamento.\\
Arrivando alla fine, intera esperienza dello stage è più che positiva, ritengo che il tirocinio sia molto importante per uno studente, permettendolo di sfruttare e valutare le proprie conoscenze acquisite e particolarmente per capire e riflettere il corso studio scelto se è quello che desiderato o meno, infatti lo stage mi ha affermato di continuare la carriera lavorativa nell'ambito di programmazione.