% !TEX encoding = UTF-8
% !TEX TS-program = pdflatex
% !TEX root = ../tesi.tex

%**************************************************************
\chapter{Analisi dei requisiti}
\label{cap:analisi dei requisiti}
%**************************************************************

\intro{In questo capitolo vengono trattati le analisi dei requisiti del modulo di front-end della web-app, con i vari casi d'uso ed un elenco dei requisiti.}\\

%**************************************************************
\section{Descrizione generale}
\subsection{Interfacce della web-app}
\subsubsection{Interfaccia menù}
L'utente può visualizzare il menù di un ristorante dopo aver scansionato il QR-code di un ristorante presente sui tavoli. Il menù è composto da un insieme di categorie, in ognuna di esse sono presenti dei piatti. I piatti sono ordinati in base al suo id che è un numero univoco dentro ogni menù.
\subsubsection{Interfaccia lista ordini}
L'utente può vedere i piatti ordinati del suo tavolo di appartenenza, dove ci sono anche i piatti ordinati dalle altre persone dello stesso tavolo, inoltre può vedere i suoi piatti personali e i piatti in arrivo.
\subsubsection{Interfaccia gestione tavolo}
In questa \gls{mascherag} utente può creare una sessione di tavolo se non appartiene a nessuna sessione, altrimenti può visualizzare il QR-code della sua sessione per fare entrare gli altri nella sua sessione di tavolo scansionando il codice.
\subsubsection{Interfaccia area personale}
Qui l'utente può effetture il login, di seguito se ha degli allergeni potrà inserire degli ingredienti nella blacklist per nascondere i piatti contenenti quegli ingredienti nella sezione menù.
\subsection{Caratteristiche degli utenti}
In questa sezione vengono descritti tutte le caratteristiche degli utenti che possono utilizzare la web-app.
\subsubsection{Utente non autenticato}
Con il termine utente non autenticato ci si riferisce ad una qualsiasi persona non autenticata nel sistema, che può sfruttare le funzionalità di base offerte dalla piattaforma, ossia:
\begin{itemize}
    \item Visualizzare il menù del ristorante;
    \item Visualizzare i singoli piatti in modalità dettaglio;
    \item Creare una sessione di tavolo;
    \item Unire ad una sessione di tavolo già esistente;
    \item Uscire dalla sessione di tavolo;
    \item Aggiungere piatti negli ordini;
    \item Aggiungere note ai piatti ordinati;
    \item Spostare ordini in arrivo;
    \item Marcare i piatti in arrivo come arrivato;
    \item Registrare nella piattaforma;
    \item Effettuare il login.
\end{itemize}
\subsubsection{Utente autenticato}
Invece, con il termine “utente autenticato” ci si riferisce ad una persona registrata nel database e che ha effettuato l'accesso nella piattaforma, la quale, oltre a sfruttare le funzionalità dell'utente non autenticato, può anche:
\begin{itemize}
    \item Aggiungere piatti nei preferiti;
    \item Rimuovere piatti dai preferiti;
    \item Dare una recensione ad un piatto;
    \item Aggiungere ingredienti non voluti;
    \item Rimuovere gli ingredienti non voluti;
    \item Visualizzare la lista dei preferiti;
    \item Effettuare logout.
\end{itemize}
%**************************************************************
\subsection{Tecnologie utilizzate}
Per sviluppare la piattaforma vengono utilizzate le seguenti tecnologie:
\begin{itemize}
    \item Angular: per la creazione dell'interfaccia utente;
    \item HTML5: per creare la struttura dell'interfaccia utente;
    \item CSS3: per lo stile dell'interfaccia, viene utilizzato la sintassi SCSS;
    \item Stoplight: per simulare le chiamate \gls{restg} \gls{apig};
    \item Angular Material: per la creazione dei componenti.
\end{itemize}
\subsection{Descrizione delle tecnologie}
\subsubsection{Angular}
Angular è un framework open source per sviluppare applicazioni web, permette di dividere l'applicazione in più componenti, grazie a questo è possibile riutilizzare lo stesso modulo in più parti della web-app, oltre a questo garantisce una maggiore manutenibilità e espandibilità.
\begin{figure}[H]
    \centering
    \includegraphics[scale=0.1]{angular.png}
    \caption{Logo di Angular}
\end{figure}
\subsubsection{HTML5}
L'HyperText Markup Language, noto come HTML, è un linguaggio di markup più popolare, utilizzato per progettare le strutture dei siti web. Viene utilizzato da Angular per creare la struttura iniziale per le varie interfacce, poi queste strutture vengono modificate da Angular per generare la pagina dinamicamente.
\begin{figure}[H]
    \centering
    \includegraphics[scale=0.5]{html css.png}
    \caption{Logo di HTML e CSS}
\end{figure}
\subsubsection{CSS3}
Il CSS, sigla di Cascading Style Sheets, è il linguaggio utilizzato per modificare il layout delle pagine web, le regole vengono applicate nel ordine in cui vengono scritte. Per il progetto è stato utilizzato una sua estensione SCSS, la quale è compatibile con tutte le versioni di CSS. Tramite la SCSS è possibile dichiarare le regole CSS in blocchi quindi ci aiuta a scrivere regole CSS in più velocemente e comprensibile.
\subsubsection{TypeScript}
Il linguaggio di programmazione utilizzato da Angular è TypeScript che è un linguaggio di programmazione open source sviluppato da Microsoft, TypeScript estende il classico JavaScript quindi qualsiasi codice scritto in JavaScript è anche eseguibile con TypeScript direttamente senza nessuna modifica. TypeScript rende molto più flessibile e flessibile JavaScript aggiungendo la firma dei metodi, classi, tipi di dato e tanto altro, grazie a queste caratteristiche utilizzando TypeScript ci ganrantisce controlli automatici, rilevando in automatico i bug prima della compilazione. Grazie a TypeScript il codice generato da Angular è eseguibile su tutti i pricipali web browser comuni come Google Chrome, Firefox, Safari, Opera, Microsoft Edge e tanti altri.
\subsubsection{Stoplight}
Stoplight è una piattaforma che offre la possibilità di progettare le API velocemente, grazie alla sua semplice interfaccia utente. Permette di collabolare con gli altri, condividendo tutte le API con le sue descrizioni e risposte in modo chiaro.
\begin{figure}[H]
    \centering
    \includegraphics[scale=0.4]{stoplight logo.png}
    \caption{Logo di Stoplight}
\end{figure}
\subsubsection{Angular Material}
Angular Material è un insieme di componenti UI già implementati, questi componenti possono essere direttamente utilizzati in Angular. Tutti i componenti offerti sono già \gls{responsiveg} e sono facili da customizzare con le proprie preferenze, come il colore dei bottoni.
\begin{figure}[H]
    \centering
    \includegraphics[scale=0.3]{angular material.png}
    \caption{Logo di Angular Material}
\end{figure}
\section{Casi d'uso}
\subsection{Introduzione}
In questa sezione verranno presentati i casi d'uso individuati durante la fase di analisi dei requisiti, i quali fanno riferimento a tutte le funzionalità che la web-app SushiLab dovrà offrire ad ogni utente che vorrà interfacciarsi con essa. Verranno elencati i casi d'uso principali con la propria descrizione, mentre i casi d'uso secondarie e le descrizioni schematiche potranno essere reperiti nell'appendice {\hyperref[cap:appendice a]{A}}.
\subsection{Attori primari}
\begin{itemize}
    \item Utente non Autenticato: utente che non ha ancora effettuato la fase di autenticazione sulla piattaforma. Può essere in possesso o meno delle credenziali per l'autenticazione. Avrà funzionalità limitate rispetto ad un utente autenticato;
    \item Utente Autenticato: utente che ha effettuato l'autenticazione alla piattaforma tramite le proprie credenziali. Ha accesso ad ogni funzionalità messa a disposizione dalla piattaforma;
    \item  Utente Generico: può essere sia un utente autenticato che un utente non autenticato.
\end{itemize}
\subsubsection{UC1 - Visualizza menù}
Nella schermata di visualizzazione menù consente di effettuare operazioni relativi all'acquisto dei piatti in base alla tipologia di utente che interagisce con la piattaforma.
Un utente generico può visualizzare un piatto in due modalità quella normale e quella in dettaglio.\\
In quella normale vengono visualizzati il nome, la descrizione, gli ingredienti e il prezzo ed è possibile modificare la quantità dei piatti contenuti negli ordini.
Nella modalità in dettaglio l'utente può visualizzare la valutazione media e opzionalmente può aggiungere una nota al piatto.\\
L'autenticazione dell'utente nel sistema permette svolgere ulteriori azioni, quali:\\
\begin{itemize}
    \item L'aggiunta di un piatto alla lista dei preferiti per tenere traccia dei piaciuti per riordinarlo in futuro;
    \item La rimozione dei piatti del menù che contengono ingredienti presenti nella propria blacklist;
    \item L'aggiunta di una recensione ad un piatto per dare consigli agli altri utenti.
\end{itemize} 

% \subsubsection{UC1.1 - Visualizza categorie}
% Un utente deve poter visualizzare le categorie dei piatti del ristorante nella sezione menù.
% \subsubsection{UC1.2 - Visualizza piatto}
% Un utente deve poter visualizzare il piatto con il nome, il numero, il prezzo, gli ingredienti, la descrizione e la quantità.
% \subsubsection{UC1.3 - Aumenta quantità}
% Un utente deve poter aumentare la quantità di un piatto fino ad un limite se il piatto è limitato.
% \subsubsection{UC1.4 - Diminuisci quantità}
% Un utente deve poter diminuire la quantità di un piatto con quantità maggiore di 1.
% \subsubsection{UC1.5 - Visualizza dettaglio piatto}
% Un utente deve poter visualizzare i dettagli del piatto con la propria valutazione, la valutazione media e le note del piatto.
% \subsubsection{UC1.6 - Nascondi dettaglio piatto}
% Un utente deve poter nascondere i dettagli di un piatto che è in modalità dettaglio.
\subsubsection{UC2 - Gestione del tavolo}
Nella schermata di gestione del tavolo consente di effettuare operazioni relativi alla sessione del tavolo. Un utente già presente in una sessione può visualizzare il codice della propria sessione e tramite questo codice è possibile generare un QR-code per fare scansionare agli altri utenti allo scopo di fare unire tutti gli utenti dello stesso tavolo nella stessa sessione. Gli utenti senza una sessione possono entrare in una sessione del tavolo tramite due creazione o unione. La creazione consente all'utente di entrare in una nuova sessione, l'unione invece permette di inserire l'utente in una sessione già presente per fare questo l'utente può digitare manualmente il codice del tavolo o scansionando direttamente il QR-code. 
% \subsubsection{UC2.1 - Generazione sessione tavolo}
% Un utente deve poter generare una sessione di tavolo quando non si trova in nessuna sessione.
% \subsubsection{UC2.2 - Unione sessione tavolo}
% Un utente deve poter unirsi ad una sessione di tavolo utilizzando il codice sessione o tramite la scansione del QR-code quando non si trova in nessuna sessione.
% \subsubsection{UC2.3 - Uscita sessione tavolo}
% Un utente deve poter uscire da una sessione di tavolo.
% \subsubsection{UC2.4 - Generazione QR-code sessione tavolo}
% Un utente presente in una sessione di tavolo deve poter generare il QR-code del tavolo.
\subsubsection{UC3 - Lista degli ordini}
La maschera della lista degli ordini può visualizzare tre sezioni, quali: ordini del tavolo, ordini personali e ordini in arrivo.
La sezione del tavolo permette di visualizzare gli ordini di tutti gli utenti della stessa sessione, questi ordini non sono ancora mandati al ristorante quindi gli utenti possono controllare tutti i piatti al fine di ordinare le portate desiderate. L'invio dell'ordine al ristorante avviene tramite la scansione del QR-code degli ordini, questo è stato implementato per evitare agli utenti malevoli di effettuare ordini anche se non sono in locale. Il ristorante una volta preso gli ordini, l'utente può spostare gli ordini in arrivo. Nella sezione degli ordini personali è possibile vedere tutti gli ordini effettuati dal l'utente stesso, includendo anche gli ordini già marcati come arrivato. Gli ordini in arrivo mostra tutte le preparazioni non ancora arrivate, un piatto una volta arrivato l'utente può marcarlo come arrivato al fine di capire quante portate mancano da arrivare e quindi ordinare altri piatti in base al proprio tempo o (pieno). I piatti nella sezione degli ordini in arrivo sono visualizzati con l'immagine e la descrizione favorendo il riconoscimento del piatto nel momento dell'arrivo.
% \subsubsection{UC3.1 - Visualizza lista ordini del tavolo}
% Un utente che si trova nella sezione di lista degli ordini deve poter visualizzare la lista degli ordini del tavolo. Composto da una lista di piatti che contiene i loro nomi e le loro quantità.
% \subsubsection{UC3.2 - Visualizza lista ordini personali}
% Un utente che si trova nella sezione di lista degli ordini deve poter visualizzare la lista degli ordini personali, che contiene solamente gli ordini che l'utente ha ordinato.
% \subsubsection{UC3.3 - Visualizza lista ordini in arrivo}
% Un utente che si trova nella sezione di lista degli ordini deve poter visualizzare la lista degli ordini in arrivo, che contiene tutti i piatti in arrivo e ognuna di essi è presente un bottone che permette di marcare il piatto come ricevuto.
% \subsubsection{UC3.1.1 - Sposta la lista degli ordini in arrivo}
% Un utente che si trova nella sezione di lista degli ordini del tavolo deve poter spostare gli ordini del tavolo nella lista degli ordini in arrivo.
% \subsubsection{UC3.1.2 - Mostra QR-code ordini}
% Un utente che si trova nella sezione di lista ordini del tavolo deve poter generare il QR-code degli ordini per poi fare scansionare dai camerieri.
% \subsubsection{UC3.2.1 - Visualizza in dettaglio lista ordini personali}
% Un utente che si trova nella sezione di lista ordini personali deve poter visualizzare la lista degli ordini personali in modalità dettaglio per poi vedere tutti i dati del piatto al fine di riconescere il proprio piatto al momento di ricezione.
% \subsubsection{UC3.3.1 - Ricezione piatto}
% Un utente che si trova nella sezione di lista degli ordini in arrivo deve poter marcare un piatto come ricevuto.
% \section{Utente Non Autenticato}
% In questa sezione vengono elencati le principali funzionalità della web-app che un utente non autenticato può usufruire.
\subsubsection{UC4 - Area personale}
La visuale dell'area personale permette all'utente di effettuare il login nella piattaforma per usufruire di ulteriori funzionalità. Gli utenti che non hanno l'account possono registrarsi su sushiLab, inoltre l'utente può recuperare la propria password inserendo la propria email utilizzata durante la registrazione. Un utente autenticato tramite il login può accedere alla propria blacklist degli ingredienti per inserire o rimuovere un ingrediente.
% \subsubsection{UC4.1 - Registrazione}
% Un utente non autenticato deve poter registrarsi alla piattaforma inserendo un'email ed una password.
% \subsubsection{UC4.2 - Login}
% Un utente già registrato deve poter effettuare il login tramite i credenziali inseriti durante la fase di registrazione.
% \subsubsection{UC4.3 - Password dimenticata}
% Un utente già registrato deve poter recuperare la password inserendo l'email utilizzato per la registrazione e inserendo il codice di sicurezza.
% \section{Utente Autenticato}
% In questa sezione vengono elencati le principali funzionalità della web-app che un utente autenticato può usufruire.
% \subsubsection{UC4.4 - Logout}
% Un utente che ha già effettuato il login deve poter effettuare il logout.
% \subsubsection{UC4.5 - Aggiungi ingredienti non voluti}
% Un utente che ha già effettuato il login deve poter aggiungere nuovi ingredienti non voluti nella propria blacklist.
% \subsubsection{UC4.6 - Rimuovi ingredienti non voluti}
% Un utente che ha già effettuato il login deve poter rimuovere un ingrediente dalla blacklist.
% \subsubsection{UC1.7 - Aggiungi preferiti}
% Un utente che ha già effettuato il login deve poter inserire un piatto nella lista dei piatti preferiti.
% \subsubsection{UC1.8 - Rimuovi preferiti}
% Un utente che ha già effettuato il login deve poter rimuovere un piatto dalla lista dei preferiti.
% \subsubsection{UC1.9 - Aggiungi recensione}
% Un utente che ha già effettuato il login deve poter inserire una rencensione ad un piatto.
% \subsubsection{UC5 - Visualizza lista preferiti}
% Un utente che ha già effettuato il login deve poter visualizzare la propria lista dei preferiti.
\section{Requisiti}
In base a quanto definito nell'analisi dei requisiti sono stati individuati una lista dei requisiti che deve essere soddisfatta durante tutto il progetto di stage. I requisiti sono individuati tramite i casi d'uso precedentemente descritti. Per ogni requisito individuato è stato assegnato un codice univoco, la descrizione del requisito e il caso d'uso riferito. Tutti i requisiti sono mostrati dentro una tabella e può essere reperita nell'appendice {\hyperref[cap:appendice b]{B}}.