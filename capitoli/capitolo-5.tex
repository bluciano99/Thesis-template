% !TEX encoding = UTF-8
% !TEX TS-program = pdflatex
% !TEX root = ../tesi.tex

%**************************************************************
\chapter{Verifica e validazione}
\label{cap:verifica}
%**************************************************************

\intro{In questa sezione vengono elencate i test di validazione effettuate e un descrizione della validazione.}\\

%**************************************************************
\section{Verifica}
Viene riportato una tabella dei test di unità sui vari componenti e sulle loro funzionalità, per ogni test viene identificato il codice del test, il requisito, una descrizione e lo stato.
\begin{center}
    \rowcolors{2}{Cyan!10}{GreenYellow!10}
    \renewcommand{\arraystretch}{1.5}
    \begin{longtable}{ |p{1cm}|p{1.5cm}|p{9cm}|p{1.5cm}|  }
        \hline
        \hline
        Test&Requisito&Descrizione &Stato \\
        \hline
        \endhead
        T1&ROF1&Il componente menù viene creato&Passato \\
        T2&ROF3&Il componente piatto viene creato&Passato \\
        T3&ROF4&La quantità viene aumentata&Passato \\
        T4&ROF5&La quantità viene diminuita&Passato \\
        T5&ROF6&Il componente piatto viene visualizzato in modalità dettaglio&Passato \\
        % T1&ROF7&L'utente può impostare la visualizzazione dei piatti del menù nella modalità normale&Passato \\
        T6&ROF8&Il componente gestione tavolo viene creato &Passato \\
        T7&ROF9&La sessione viene generata&Passato\\
        T8&ROF10&L'utente viene unito ad una sessione&Passato \\
        T9&ROF11&L'utente esce dalla sessione&Passato\\
        T10&ROF12&Il QR-code del tavolo viene generato&Passato\\
        T11&ROF13&Il componente gestione lista ordini viene creato&Passato\\
        % T1&ROF14&All'utetne viene mostrato la lista ordini del tavolo&Passato\\
        % T1&ROF15&All'utetne viene mostrato la lista ordini personali &Passato\\
        % T1&ROF16&All'utetne viene mostrato la lista ordini in arrivo&Passato\\
        T12&ROF17&La lista ordini del tavolo viene spostato in arrivo&Passato\\
        T13&ROF18&Il QR-code della lista ordini viene generato &Passato\\
        % T1&ROF19&L'utente può impostare la visualizzazione dei piatti della lista ordini personali in modalità dettaglio&Passato\\
        T14&ROF20&Il piatto delle lista ordini in arrivo viene marcato come arrivato&Passato\\
        T15&ROF21&Il componente area personale viene creato&Passato\\
        T16&ROF22&Il componente register viene creato&Passato\\
        T17&ROF22&I campi sbagliati del form di registrazione vengono evidenziati&Passato\\
        T18&ROF22&Il submit del form deve registrare l'utente&Passato\\
        T19&ROF23&Il componente login viene creato&Passato\\
        T20&ROF23&I campi sbagliati del form di login vengono evidenziati &Passato\\
        T21&ROF23&Il submit del form deve effettuare il login dell'utente &Passato\\
        T22&ROF24&Il componente forgot viene creato  &Passato\\
        T23&ROF24&I campi sbagliati del form di password dimenticata vengono evidenziati &Passato\\
        T24&ROF24&Il submit del form deve reimpostare la password&Passato\\
        T25&ROF25&Il componente logout viene creato&Passato\\
        T26&ROF25&Il click sul componente logout deve effetture il logout dell'utente&Passato\\
        T27&ROF26&Il componente blacklist viene creato&Passato\\
        T28&ROF26&L'ingrediente viene inserito nella blacklist&Passato\\
        T29&ROF27&L'ingrediente viene rimosso dalla blacklist&Passato\\
        T30&ROF28&Il piatto viene aggiunto nella lista dei preferiti&Passato\\
        T31&ROF29&Il piatto viene rimosso dalla lista dei preferiti&Passato\\
        T32&ROF30&Il componente star viene creato&Passato\\
        T34&ROF30&La recensione viene impostato correttamente&Passato\\
        T35&ROF31&Il componente lista preferiti viene creato&Passato\\
\hline
\caption{\label{tab:tabella dei test sui requisiti}Tabella dei test sui requisiti.}
\end{longtable}
\end{center}
\section{Validazione e collaudo}
Nell'ultima settimana dello stage ho controllato tutte le funzionalità implementate durante intero progetto, dove ho testato tutte le attività che sono svolgibili da un untente generale. Infine è stato mostrato il prodotto finale del progetto sushiLab al tutor, facendo un demo completo di tutte le funzionalità della web-app.